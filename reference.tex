\input{preamble.tex}

\begin{document}

\def\title{Tecnológico de Monterrey Campus Querétaro}
.\\[0.2cm]
%\centering{\includegraphics[width=5.5cm]{img/vasito.jpg}}
\tableofcontents\newpage

\section{Data structures}
\subsection{Segment tree}
\cppfile{data_structures/segment_tree.cpp}

\section{Matemáticas}
\subsection{Matrices}
Multiplicación de matrices
\cppfile{matematicas/multiplicacion_matrices.cpp}
\subsection{Divisores}
\cppfile{matematicas/divisores.cpp}
El siguiente código sirve para implementar la función \lstinline{calcular_divisores}. El algoritmo tiene complejidad de $O(\sqrt n)$.
\cppfile{matematicas/calcular_divisores.cpp}
\subsection{Criba de primos}
La siguiente criba les servirá para calcular los primeros 78,499 primos (todos los primos menores o iguales a un millón). Si quieren saber si un número es primo, lo pueden consultar de manera constante checando el arreglo \lstinline{es_primo}. Antes de usar los arreglos \lstinline{es_primo} y \lstinline{primos}, hay que llamar a la función \lstinline{criba_primos}.

\end{document}
